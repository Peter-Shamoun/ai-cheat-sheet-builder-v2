 



\documentclass{article}
\usepackage[landscape]{geometry}
\usepackage{url}
\usepackage{multicol}
\usepackage{amsmath}
\usepackage{esint}
\usepackage{amsfonts}
\usepackage{tikz}
\usetikzlibrary{decorations.pathmorphing}
\usepackage{amsmath,amssymb}

\usepackage{colortbl}
\usepackage{xcolor}
\usepackage{mathtools}
\usepackage{amsmath,amssymb}
\usepackage{enumitem}
\makeatletter

\newcommand*\bigcdot{\mathpalette\bigcdot@{.5}}
\newcommand*\bigcdot@[2]{\mathbin{\vcenter{\hbox{\scalebox{#2}{$\m@th#1\bullet$}}}}}
\makeatother

\title{Title Cheat Sheet}
\usepackage[brazilian]{babel}
\usepackage[utf8]{inputenc}

\advance\topmargin-.8in
\advance\textheight3in
\advance\textwidth3in
\advance\oddsidemargin-1.5in
\advance\evensidemargin-1.5in
\parindent0pt
\parskip2pt
\newcommand{\hr}{\centerline{\rule{3.5in}{1pt}}}
%\colorbox[HTML]{e4e4e4}{\makebox[\textwidth-2\fboxsep][l]{texto}
\begin{document}

\begin{center}{\huge{\textbf{Title Cheat Sheet}}}\\
\end{center}
\begin{multicols*}{3}

\tikzstyle{mybox} = [draw=black, fill=white, very thick,
    rectangle, rounded corners, inner sep=10pt, inner ysep=10pt]
\tikzstyle{fancytitle} =[fill=black, text=white, font=\bfseries]

%------------ Relational Databases ---------------
\begin{tikzpicture}
\node [mybox] (box){%
    \begin{minipage}{0.3\textwidth}
   \begin{itemize}
\item A relation in a database corresponds to a table. 
\item Rows are called tuples; columns are attributes.
\item Primary Key: uniquely identifies a tuple.
\item Foreign Key: used to establish a relationship between tables.
\item The order of rows is not significant.
\item Physical and logical data independence allows schema change without affecting apps.
\end{itemize}
   
\end{minipage}
};
%------------ Relational Databases  Header ---------------------
\node[fancytitle, right=10pt] at (box.north west) {Relational Databases};
\end{tikzpicture}

%------------ Structured Query Language (SQL)  ---------------
\begin{tikzpicture}
\node [mybox] (box) {%
    \begin{minipage}{0.3\textwidth}
       \begin{itemize}
\item SQL is used for data querying in relational databases.
\item DDL includes CREATE, ALTER, DROP tables.
\item DML: SELECT for retrieval, INSERT, UPDATE, DELETE.
\item Fundamental in data science for data transformation.
\item Basic form: \texttt{SELECT columns FROM table WHERE condition;}.
\end{itemize}
    \end{minipage}
};
%------------ Structured Query Language (SQL)  Header ---------------------
\node[fancytitle, right=10pt] at (box.north west) {Structured Query Language (SQL)};
\end{tikzpicture}

%------------ SQLite Database ---------------
\begin{tikzpicture}
\node [mybox] (box){%
    \begin{minipage}{0.3\textwidth}
    \begin{itemize}
\item SQLite is a lightweight, file-based DB.
\item Useful for small apps, testing, embedded systems.
\item Non-commercial; supports SQL standard.
\item Offers limited concurrency features.
\item ACID compliant via rollback journals.
\end{itemize}
    \end{minipage}
};
%------------ SQLite Database Header ---------------------
\node[fancytitle, right=10pt] at (box.north west) {SQLite Database};
\end{tikzpicture}

%------------ DB Browser for SQLite ---------------
\begin{tikzpicture}
\node [mybox] (box){%
    \begin{minipage}{0.3\textwidth}
   \begin{itemize}
\item Tool for managing SQLite DBs without scripting.
\item Easily run queries, create/modify tables.
\item Import/export data in CSV format.
\item Intuitive GUI, catering to various skill levels.
\end{itemize}
    \end{minipage}
};
%------------ DB Browser for SQLite Header ---------------------
\node[fancytitle, right=10pt] at (box.north west) {DB Browser for SQLite};
\end{tikzpicture}
%------------ Selection and Projection in SQL ---------------------
\begin{tikzpicture}
\node [mybox] (box){%
    \begin{minipage}{0.3\textwidth}
    	\begin{itemize}
\item \textbf{SELECT} chooses specific columns from a table.
\item \textbf{WHERE} clause filters rows based on conditions.
\item \textbf{Projection}: retrieve specific attributes.
\item Conditions without WHERE result in full table data.
\end{itemize}
    \end{minipage}
};
%------------ Selection and Projection in SQL Header ---------------------
\node[fancytitle, right=10pt] at (box.north west) {Selection and Projection in SQL};
\end{tikzpicture}

%------------ Joins in SQL ---------------
\begin{tikzpicture}
\node [mybox] (box){%
    \begin{minipage}{0.3\textwidth}  
   \begin{itemize}
\item Joins combine rows from two or more tables.
\item \textbf{INNER JOIN}: returns rows with matching keys.
\item \textbf{LEFT/RIGHT JOIN}: includes all from one table, matches from another.
\item \textbf{FULL JOIN}: unions all matched and unmatched rows.
\end{itemize}
    \end{minipage}
};
%------------ Joins in SQL Header ---------------------
\node[fancytitle, right=10pt] at (box.north west) {Joins in SQL};
\end{tikzpicture}

%------------ Minimum Spanning Trees (MSTs) ---------------
\begin{tikzpicture}
\node [mybox] (box){%
    \begin{minipage}{0.3\textwidth}
    \begin{itemize}
\item Prim's algorithm: starts at a node, expands by minimum edge.
\item Kruskal's: sorts edges, adds smallest edge not forming cycles.
\item MST edge weight sum is minimal.
\end{itemize}
    \end{minipage}
};
%------------ Minimum Spanning Trees (MSTs) Header ---------------------
\node[fancytitle, right=10pt] at (box.north west) {Minimum Spanning Trees (MSTs)};
\end{tikzpicture}


%------------ Graph Weight Modification and MSTs ---------------
\begin{tikzpicture}
\node [mybox] (box){%
    \begin{minipage}{0.3\textwidth}
\begin{itemize}
\item Doubling edge weights scales MST weights.
\item Adding a constant does not change MST itself.
\end{itemize}
    \end{minipage}
};
%------------ Graph Weight Modification and MSTs Header ---------------------
\node[fancytitle, right=10pt] at (box.north west) {Graph Weight Modification and MSTs};
\end{tikzpicture}
\
%------------ Shortest Path Trees ---------------
\begin{tikzpicture}
\node [mybox] (box){%
    \begin{minipage}{0.3\textwidth}
    \begin{itemize}
\item Shortest path trees prioritize path cost.
\item MST and shortest trees differ in objectives.
\end{itemize}
    \end{minipage}
};
%------------ Shortest Path Trees Header ---------------------
\node[fancytitle, right=10pt] at (box.north west) {Shortest Path Trees};
\end{tikzpicture}
%------------ Relational Databases0 ---------------------
\begin{tikzpicture}
\node [mybox] (box){%
    \begin{minipage}{0.3\textwidth}
	\begin{itemize}
\item A relation in a database corresponds to a table. 
\item Rows are called tuples; columns are attributes.
\item Primary Key: uniquely identifies a tuple.
\item Foreign Key: used to establish a relationship between tables.
\item The order of rows is not significant.
\item Physical and logical data independence allows schema change without affecting apps.
\end{itemize}0
	\end{minipage}
};
%------------ Relational Databases0 Header ---------------------
\node[fancytitle, right=10pt] at (box.north west) {Relational Databases0};
\end{tikzpicture}
\\
\\
\\
\\

%------------ Relational Databases1 ---------------------
\begin{tikzpicture}
\node [mybox] (box){%
    \begin{minipage}{0.3\textwidth}
   \begin{itemize}
\item A relation in a database corresponds to a table. 
\item Rows are called tuples; columns are attributes.
\item Primary Key: uniquely identifies a tuple.
\item Foreign Key: used to establish a relationship between tables.
\item The order of rows is not significant.
\item Physical and logical data independence allows schema change without affecting apps.
\end{itemize}1
	\end{minipage}
};
%------------ Relational Databases1 Header ---------------------
\node[fancytitle, right=10pt] at (box.north west) {Relational Databases1};
\end{tikzpicture}

%------------ Relational Databases2 ---------------
\begin{tikzpicture}
\node [mybox] (box){%
    \begin{minipage}{0.3\textwidth}
    \begin{itemize}
\item A relation in a database corresponds to a table. 
\item Rows are called tuples; columns are attributes.
\item Primary Key: uniquely identifies a tuple.
\item Foreign Key: used to establish a relationship between tables.
\item The order of rows is not significant.
\item Physical and logical data independence allows schema change without affecting apps.
\end{itemize}2
    \end{minipage}
};
%------------ Relational Databases2 Header ---------------------
\node[fancytitle, right=10pt] at (box.north west) {Relational Databases2};
\end{tikzpicture}
\end{multicols*}
\end{document}