 
\documentclass{article}
\usepackage[landscape]{geometry}
\usepackage{url}
\usepackage{multicol}
\usepackage{amsmath}
\usepackage{esint}
\usepackage{amsfonts}
\usepackage{tikz}
\usetikzlibrary{decorations.pathmorphing}
\usepackage{amsmath,amssymb}

\usepackage{colortbl}
\usepackage{xcolor}
\usepackage{mathtools}
\usepackage{amsmath,amssymb}
\usepackage{enumitem}
\makeatletter

\newcommand*\bigcdot{\mathpalette\bigcdot@{.5}}
\newcommand*\bigcdot@[2]{\mathbin{\vcenter{\hbox{\scalebox{#2}{$\m@th#1\bullet$}}}}}
\makeatother

\title{Title Cheat Sheet}
\usepackage[brazilian]{babel}
\usepackage[utf8]{inputenc}

\advance\topmargin-.8in
\advance\textheight3in
\advance\textwidth3in
\advance\oddsidemargin-1.5in
\advance\evensidemargin-1.5in
\parindent0pt
\parskip2pt
\newcommand{\hr}{\centerline{\rule{3.5in}{1pt}}}
%\colorbox[HTML]{e4e4e4}{\makebox[\textwidth-2\fboxsep][l]{texto}
\begin{document}

\begin{center}{\huge{\textbf{Title Cheat Sheet}}}\\
\end{center}
\begin{multicols*}{3}

\tikzstyle{mybox} = [draw=black, fill=white, very thick,
    rectangle, rounded corners, inner sep=10pt, inner ysep=10pt]
\tikzstyle{fancytitle} =[fill=black, text=white, font=\bfseries]

%------------ Relational Database Concepts ---------------
\begin{tikzpicture}
\node [mybox] (box){%
    \begin{minipage}{0.3\textwidth}
   
        \textbf{Relational Database:} A relational database organizes data into tables (relations) consisting of rows and columns, where each table represents a different entity.\\
        \textbf{Primary Key:} A primary key is a unique identifier for each record in a table, ensuring that no two rows have the same primary key value.\\
        \textbf{Foreign Key:} A foreign key is a column or set of columns in one table that uniquely identifies a row of another table, establishing a relationship between the two tables.\\
        \textbf{Normalization: 1NF:} A table is in First Normal Form (1NF) if all its columns contain atomic, indivisible values and each column contains values of a single type.\\
        \textbf{Database Relations:} Relations in a database refer to the logical connections between tables, often established through primary and foreign keys to maintain data integrity.\\
            
\end{minipage}
};
%------------ Relational Database Concepts  Header ---------------------
\node[fancytitle, right=10pt] at (box.north west) {Relational Database Concepts};
\end{tikzpicture}

%------------ SQL Basics  ---------------
\begin{tikzpicture}
\node [mybox] (box) {%
    \begin{minipage}{0.3\textwidth}
       \textbf{Structured Query Language (SQL):} SQL is a domain-specific language used in programming and designed for managing data held in a relational database management system (RDBMS).\\
        \textbf{Data Definition Language (DDL):} DDL includes SQL commands such as CREATE, ALTER, and DROP, which are used to define or modify database structures.\\
        \textbf{Data Manipulation Language (DML):} DML consists of SQL commands like SELECT, INSERT, UPDATE, and DELETE, which are used to retrieve and manipulate data in a database.\\
        \textbf{SQL Statements:} SQL statements are used to perform tasks such as updating data on a database, or retrieving data from a database, and include SELECT, INSERT, UPDATE, DELETE, etc.\\
        \textbf{SQL Joins:} Joins are SQL operations that allow you to combine rows from two or more tables based on a related column between them, such as INNER JOIN, LEFT JOIN, RIGHT JOIN, and FULL JOIN.\\
    \end{minipage}
};
%------------ SQL Basics  Header ---------------------
\node[fancytitle, right=10pt] at (box.north west) {SQL Basics};
\end{tikzpicture}

%------------ SQLite ---------------
\begin{tikzpicture}
\node [mybox] (box){%
    \begin{minipage}{0.3\textwidth}
    \textbf{SQLite Architecture:} SQLite is a file-based database engine that stores the entire database in a single file, making it lightweight and easy to deploy.\\
        \textbf{SQL92 Standard Compliance:} SQLite supports most of the SQL92 standard, providing a wide range of SQL features for database management.\\
        \textbf{Atomic Commit and Rollback:} SQLite ensures data integrity through atomic commit and rollback capabilities, allowing transactions to be completed fully or not at all.\\
        \textbf{Data Types in SQLite:} SQLite uses dynamic typing, allowing flexibility in storing data types such as INTEGER, REAL, TEXT, BLOB, and NULL.\\
        \textbf{Concurrency Control:} SQLite uses a locking mechanism to manage concurrency, allowing multiple readers or a single writer at any time.\\
    \end{minipage}
};
%------------ SQLite Header ---------------------
\node[fancytitle, right=10pt] at (box.north west) {SQLite};
\end{tikzpicture}

%------------ Minimum Spanning Tree Algorithms ---------------
\begin{tikzpicture}
\node [mybox] (box){%
    \begin{minipage}{0.3\textwidth}
        \textbf{Minimum Spanning Tree (MST) Definition:} An MST of a graph $G$ is a subset of edges that connects all vertices with the minimum possible total edge weight.\\
        \textbf{Prim's Algorithm:} Prim's algorithm starts with a single vertex and grows the MST by adding the cheapest edge from the tree to a vertex not yet in the tree.\\
        \textbf{Kruskal's Algorithm:} Kruskal's algorithm builds the MST by sorting all edges and adding them one by one, ensuring no cycles are formed, until all vertices are connected.\\
        \textbf{Edge Weights in MST:} Edge weights determine the selection of edges in MST algorithms, where the goal is to minimize the sum of the weights of the edges in the tree.\\
        \textbf{Cycle Prevention in Kruskal's Algorithm:} Kruskal's algorithm uses a union-find data structure to efficiently check for cycles when adding edges to the MST.\\    \end{minipage}
};
%------------ Minimum Spanning Tree Algorithms Header ---------------------
\node[fancytitle, right=10pt] at (box.north west) {Minimum Spanning Tree Algorithms};
\end{tikzpicture}
%------------ Graph Theoretical Concepts ---------------------
\begin{tikzpicture}
\node [mybox] (box){%
    \begin{minipage}{0.3\textwidth}
    	\textbf{Undirected Graph:} An undirected graph $G = (V, E)$ consists of a set of vertices $V$ and a set of edges $E$, where each edge is an unordered pair of vertices.\\
        \textbf{Weighted Graph:} In a weighted graph, each edge $(u, v) \in E$ has an associated weight $w(u, v)$, representing the cost or distance between vertices $u$ and $v$.\\
        \textbf{Vertex:} A vertex, also known as a node, is a fundamental part of a graph, representing an entity or a point where edges meet.\\
        \textbf{Edge:} An edge in a graph is a connection between two vertices, and in an undirected graph, it is represented as an unordered pair $(u, v)$.\\
        \textbf{Graph Connectivity:} A graph is connected if there is a path between every pair of vertices; otherwise, it is disconnected.\\
    \end{minipage}
};
%------------ Graph Theoretical Concepts Header ---------------------
\node[fancytitle, right=10pt] at (box.north west) {Graph Theoretical Concepts};
\end{tikzpicture}

%------------ Standard Normal Distribution ---------------
\begin{tikzpicture}
\node [mybox] (box){%
    \begin{minipage}{0.3\textwidth}  
   \textbf{Standard Normal Distribution:} The standard normal distribution, denoted as $Z \sim N(0,1)$, is a normal distribution with a mean of 0 and a standard deviation of 1.\\
        \textbf{Probability of Z:} The probability $P(Z > z)$ represents the area under the standard normal curve to the right of a given $z$-score.\\
        \textbf{Area Under the Curve:} The area under the $N(0,1)$ distribution curve between two points gives the probability that $Z$ falls within that interval.\\
        \textbf{Z-Score Calculation:} A $z$-score is calculated as $z = \frac{X - \mu}{\sigma}$, where $X$ is a value from the dataset, $\mu$ is the mean, and $\sigma$ is the standard deviation.\\
        \textbf{Applications of Standard Normal Distribution:} The standard normal distribution is used to find probabilities and percentiles for normally distributed data by converting to $z$-scores.\\
    \end{minipage}
};
%------------ Standard Normal Distribution Header ---------------------
\node[fancytitle, right=10pt] at (box.north west) {Standard Normal Distribution};
\end{tikzpicture}

%------------ Hypothesis Testing in Statistics ---------------
\begin{tikzpicture}
\node [mybox] (box){%
    \begin{minipage}{0.3\textwidth}
    \textbf{Null Hypothesis ($H_0$):} The null hypothesis $H_0$ is a statement that there is no effect or no difference, and it is assumed true until evidence indicates otherwise.\\
        \textbf{Alternative Hypothesis ($H_a$):} The alternative hypothesis $H_a$ is a statement that indicates the presence of an effect or a difference, opposing the null hypothesis.\\
        \textbf{Population Parameter:} A population parameter is a numerical characteristic of a population, such as a mean ($\mu$) or standard deviation ($\sigma$), that is estimated using sample data.\\
        \textbf{Sample Mean ($\bar{x}$):} The sample mean $\bar{x}$ is the average of a set of sample data, calculated as $\bar{x} = \frac{1}{n} \sum_{i=1}^{n} x_i$, where $n$ is the sample size.\\
        \textbf{Significance Level ($\alpha$):} The significance level $\alpha$ is the probability of rejecting the null hypothesis when it is true, commonly set at 0.05 or 5\%.\\
    \end{minipage}
};
%------------ Hypothesis Testing in Statistics Header ---------------------
\node[fancytitle, right=10pt] at (box.north west) {Hypothesis Testing in Statistics};
\end{tikzpicture}


%------------ Type I and Type II Errors ---------------
\begin{tikzpicture}
\node [mybox] (box){%
    \begin{minipage}{0.3\textwidth}
\textbf{Type I Error:} A Type I error occurs when we reject a true null hypothesis, with probability denoted by $\alpha$, the significance level.\\
        \textbf{Type II Error:} A Type II error occurs when we fail to reject a false null hypothesis, with probability denoted by $\beta$.\\
        \textbf{Significance Level:} The significance level $\alpha$ is the threshold for rejecting the null hypothesis, often set at 0.05 or 0.01.\\
        \textbf{Power of a Test:} The power of a test is $1 - \beta$, representing the probability of correctly rejecting a false null hypothesis.\\
        \textbf{Trade-off Between Type I and Type II Errors:} Reducing $\alpha$ to decrease Type I error increases $\beta$, thus increasing the chance of a Type II error.\\
    \end{minipage}
};
%------------ Type I and Type II Errors Header ---------------------
\node[fancytitle, right=10pt] at (box.north west) {Type I and Type II Errors};
\end{tikzpicture}
\
%------------ Confidence Intervals and Rejection Regions ---------------
\begin{tikzpicture}
\node [mybox] (box){%
    \begin{minipage}{0.3\textwidth}
        \textbf{Confidence Intervals:} A confidence interval for a parameter $\theta$ is an interval $[L, U]$ such that $P(L \leq \theta \leq U) = 1 - \alpha$, where $\alpha$ is the significance level.\\
        \textbf{Rejection Regions:} The rejection region is the set of all values of the test statistic for which the null hypothesis $H_0$ is rejected in favor of the alternative hypothesis $H_1$.\\
        \textbf{Standard Error:} The standard error (SE) is the standard deviation of the sampling distribution of a statistic, often estimated as $SE = \frac{s}{\sqrt{n}}$, where $s$ is the sample standard deviation.\\
        \textbf{Critical Value:} The critical value is the threshold value that the test statistic must exceed for the null hypothesis to be rejected, typically found from a statistical distribution table.\\
        \textbf{Significance Level:} The significance level $\alpha$ is the probability of rejecting the null hypothesis when it is true, commonly set at 0.05 or 0.01 in hypothesis testing.\\
    \end{minipage}
};
%------------ Confidence Intervals and Rejection Regions Header ---------------------
\node[fancytitle, right=10pt] at (box.north west) {Confidence Intervals and Rejection Regions};
\end{tikzpicture}
%------------ Algorithmic Complexity and Optimization---------------------
\begin{tikzpicture}
\node [mybox] (box){%
    \begin{minipage}{0.3\textwidth}
	    \textbf{Time Complexity:} Time complexity measures the amount of time an algorithm takes to complete as a function of the length of the input, commonly expressed using Big O notation, e.g., $O(n)$ for linear time.\\
        \textbf{Optimal Algorithm:} An optimal algorithm is one that solves a problem in the least possible time or space complexity, often serving as a benchmark for evaluating other algorithms.\\
        \textbf{Linear Time Algorithms:} An algorithm runs in linear time, $O(n)$, if the time to complete is directly proportional to the size of the input data set.\\
        \textbf{Quadratic Time Algorithms:} Quadratic time complexity, $O(n^2)$, indicates that the time taken is proportional to the square of the size of the input data set, often seen in nested loop scenarios.\\
        \textbf{Algorithmic Optimization:} Algorithmic optimization involves improving an algorithm to reduce its time or space complexity, often by eliminating redundant operations or using more efficient data structures.\\
	\end{minipage}
};
%------------ Algorithmic Complexity and Optimization Header ---------------------
\node[fancytitle, right=10pt] at (box.north west) {Algorithmic Complexity and Optimization};
\end{tikzpicture}

%------------ SQL Queries ---------------------
\begin{tikzpicture}
\node [mybox] (box){%
    \begin{minipage}{0.3\textwidth}
        \textbf{FROM Statement:} The FROM statement specifies the table from which to retrieve or delete data, forming the basis of the SQL query.\\
        \textbf{WHERE Clause:} The WHERE clause filters records based on specified conditions, allowing for precise data retrieval in SQL queries.\\
        \textbf{SELECT Statement:} The SELECT statement is used to query the database and retrieve data, specifying columns to be displayed.\\
        \textbf{Joins in SQL:} Joins in SQL are used to combine rows from two or more tables based on a related column, enabling complex queries across multiple datasets.\\
        \textbf{SQL Query Optimization:} Optimizing SQL queries involves using indexes, avoiding unnecessary columns in SELECT, and minimizing subqueries to improve performance.\\
	\end{minipage}
};
%------------ SQL Queries Header ---------------------
\node[fancytitle, right=10pt] at (box.north west) {SQL Queries};
\end{tikzpicture}

%------------ Experimental Design in Statistics ---------------
\begin{tikzpicture}
\node [mybox] (box){%
    \begin{minipage}{0.3\textwidth}
        \textbf{Sample Size:} The sample size $n$ affects the precision of estimates and the power of a statistical test, with larger samples providing more reliable results.\\
        \textbf{Power of a Test:} The power of a test, $1 - \beta$, is the probability of correctly rejecting a false null hypothesis, and it increases with larger sample sizes and effect sizes.\\
        \textbf{Significance Level:} The significance level $\alpha$ is the probability of rejecting the null hypothesis when it is true, commonly set at 0.05 for a 5% risk of Type I error.\\
        \textbf{Variance:} Variance measures the dispersion of data points around the mean, and in experimental design, controlling variance is crucial for detecting true effects.\\
        \textbf{Experimental Design:} Experimental design involves planning how to collect data efficiently and effectively to answer research questions, often using randomization to reduce bias.\\
    \end{minipage}
};
%------------ Experimental Design in Statistics Header ---------------------
\node[fancytitle, right=10pt] at (box.north west) {Experimental Design in Statistics};
\end{tikzpicture}
\end{multicols*}
\end{document}

